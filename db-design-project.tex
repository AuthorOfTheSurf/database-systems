\documentclass[11pt, a4paper]{report}
\usepackage{fullpage}
\title{Modeling an Online Music Business}
\author{Zane Kansil\\ Loyola Marymount University\\ Database Systems}

% Allow commands
\providecommand\phantomsection{}

% Custom ToC labeling
\renewcommand\thechapter{\Roman{chapter}}
\renewcommand\thesection{\thechapter.\arabic{section}}

% Custom ToC spacing between chapter number and
% chapter name
\usepackage{tocloft}
\setlength{\cftchapnumwidth}{3em}
\setlength{\cftsecnumwidth}{3.5em}
\setlength{\cftsubsecnumwidth}{4em}

\begin{document}

%---    Title page    ---%
\clearpage
\phantomsection
\addcontentsline{toc}{chapter}{%
    \protect\numberline{I}Title Page}
\maketitle

%---    ToC    ---%
\clearpage
\phantomsection
\addcontentsline{toc}{chapter}{%
    \protect\numberline{II}Contents}
\tableofcontents
% The title page and ToC are difficult to handle,
% we set the page and chapter counter after the
% ToC is generated so that the rest of the document
% can continue on from there.
\setcounter{page}{2}
\setcounter{chapter}{2}

%---    Description of the Enterprise    ---%
\chapter{Description of the Enterprise}
A friend of mine intends to put out an all­-purpose site for his musical career. The site will be a hub for his online business and allow him to track sales and interactions with his fans.\\

The site will serve as a place to feature embedded music and video players for John's music. It will be necessary to monitor this hosted media in terms of listens and views. Some other metrics John is interested in are on­-site plays (not a redirect), and redirects to his Soundcloud and YouTube accounts from their embedded players on his site. Videos and Songs are tracked separately due to their differing properties.\\

There is a sales aspect to the site, merchandise will also be sold under the same domain. Some
merchandise will be physical, such as hats, headbands, wristbands and stickers. Other merchandise
will be digital, such as a donation to download a mixtape (a non-album collection of John's music). Physical merchandise needs to be shipped, so appropriate data such as \texttt{shipping\_status} and \texttt{destination} will need to be tracked. With physical merchandise, quantity and availability must be tracked. Digital merchandise is easier to manage as it is transmitted and there is less customer data to collect. Digital merchandise will also be
infinitely available if listed, so there is no quantity­ related data to track.\\

JohnDB records data for Physical and Digital consumers separately.\\

As the main operator of his business, John wants to track of his sales and revenue. The most simple way to track this would be through a table of sales records. These records would detail everything necessary about the sale. A sale would correspond to a single type of product. If multiple products were purchased in the same transaction, then multiple sale entries would be logged.\\

\clearpage
\section{Ten sample Questions John would ask}
\begin{enumerate}
    \item How many video plays today?
    \item How many redirects to Soundcloud from embedded music players?
    \item What is the redirect rate for videos?
    \item Is ``Black Silk Hooded Sweatshirt'' sold out?
    \item How many ``Red Summer Beanie'' items were sold in October 2014?
    \item What physical goods are currently frozen? (sales prevented)
    \item How many orders to I have to fill to the US?
    \item What precentage of my digital consumers are from outside the US?
    \item What is the average monthly revenue over the past six months?
    \item What products have garnered zero sales in the past 14 days? 
\end{enumerate}

%---    Definition of Environment    ---%
\chapter{Definition of Environment}

\section{Input and Report forms}
\section{Assumptions}
\section{User-oriented data dictionary}
\section{Cross-reference table}

%---    Enterprise Database Design    ---%
\chapter{Enterprise Database Design}

\section{Logical model of the Enterprise}
\subsection{List of Entities and Attributes}
\subsection{List of Relationships and Attributes}
\subsection{Entity-Relationship diagram of the Enterprise}

\section{Conceptual model of the enterprise}
\section{Table dictionary}
\section{Attribute dictionary}

%---    Database and Query Definition    ---%
\chapter{Database and Query Definition}

\section{Database Definition}
    SQL DDL for your database objects.
\section{Database Queries}
    English version of 10+ database queries, and the SQL DML for each database query
\section{Review sign-off sheet}
\section{Design Tradeoffs and Limitations}
    Discussion of the limitations of your design.

%---    Database Integrity and Security    ---%
\chapter{Database Integrity and Security}
\section{Functional Dependencies}
    A list of the functional dependencies that hold on your database.
\section{Adjustments for Normalization}
    An explanation of the changes needed to normalize your database.
\section{Integrity and Security}
    A list (in English) of the integrity and security constraints which are to hold on your database.

%---    Implementation Notes    ---%
\chapter{Implementation Notes}
\section{Indices}
    A list of the indices used by your database, with a justification for each.
\section{Data}
    The data used to populate your database.
\section{Query Trace}
    A trace of the execution of each of your queries.
\section{Implementation Assessment}
    An assessment of how smoothly your implementation went

%---    Lessons Learned    ---%
\chapter{Lessons Learned}

\end{document}
